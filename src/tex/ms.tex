% Define document class
\documentclass[twocolumn]{aastex631}
\usepackage{showyourwork}
\usepackage[ruled,vlined,linesnumbered]{algorithm2e}
\usepackage{algorithmic}
\usepackage{color}
%\usepackage[dvipsnames]{xcolor}

\definecolor{rb4}{HTML}{27408B}
\newcommand{\kw}[1]{{\color{rb4}[KW: #1 ]}}
\definecolor{pyRed}{RGB}{214, 39, 40}
\newcommand{\te}[1]{\textbf{\color{pyRed}(TE: #1)}}
\newcommand{\mi}[1]{\textsf{\color{teal}[\textbf{MI:} #1]}}

\newcommand{\flatiron}{\affiliation{Center for Computational Astrophysics,
Flatiron Institute, New York, NY 10010, USA}}
\newcommand{\jhu}{\affiliation{William H. Miller III Department of Physics and Astronomy, Johns Hopkins University, Baltimore, Maryland 21218, USA}}

\SetKwInput{Parameters}{Parameters}
\SetKwInput{Variables}{Variables}

% Begin!
\begin{document}

% Title
\title{Fast gravitational wave parameter estimation without compromises}
% Title Options
% \title{Ripples in the Flow: \\ Fast, General Parameter Estimation with Normalizing Flows and Differentiable Waveforms}

% Author list
\author{Kaze W. K. Wong}
\email{kwong@flatironinstitute.org}
\flatiron

\author{Maximiliano Isi}
\flatiron

\author{Thomas D. P. Edwards}
\jhu

% Abstract with filler text
\begin{abstract}
We present a lightweight, flexible, and high-performance framework for inferring
the properties of gravitational-wave events. By combining likelihood
heterodyning, automatically-differentiable and accelerator-compatible waveforms,
and gradient-based Markov chain Monte Carlo (MCMC) sampling enhanced by
normalizing flows, we achieve full Bayesian parameter estimation for real events
like GW150914 and GW170817 within a minute of sampling time.  Our framework does
not require pre-training or explicit reparameterizations and can be generalized
to handle higher dimensional problems. We present the details of our
implementation and discuss trade-offs and future developments in the context of
other proposed strategies for real-time parameter estimation. Our code for
generating the manuscript and running the analysis is publicly available on
GitHub.
\end{abstract}

% Main body with filler text
\section{Introduction}
\label{sec:intro}

% Brief description of PE
Parameter estimation (PE) underpins all of gravitational-wave physics and
astrophysics, and is one of the most commonly performed tasks in
gravitational-wave (GW) data analysis \cite{Christensen:2022bxb,
2019PASA...36...10T}. The central goal of PE is to infer the parameters of a
particular GW source given the strain data recorded by instruments like LIGO
\cite{LIGOScientific:2014pky}, Virgo \cite{VIRGO:2014yos} and KAGRA
\cite{KAGRA:2020tym}.  In the standard compact binary coalescence (CBC)
scenario, this could mean inferring intrinsic parameters such the masses and
spins of the compact objects, as well as extrinsic parameters such as their sky
localization and distance from Earth. PE is also applied to test general
relativity (GR) and constrain deviations away from its predictions in observed
data
\cite{LIGOScientific:2016lio,LIGOScientific:2018dkp,LIGOScientific:2021sio}. PE
is a crucial step in GW science, since it translates characteristics of the
strain data into astrophysically relevant quantities that can be used to
constrain astrophysical phenomena, including informing theories of binary
evolution \cite{LIGOScientific:2021psn} and measuring the properties of nuclear
matter.

% Old codes have trouble handling next gen data
% The PE results in the most recent catalog of GW events are produced by several
% community-developed PE codes
% \mi{what catalog do you mean? The LVK doesn't use PyCBC Inference}
There exist a number of prominent, community-developed PE codes, including
\textsc{LALInference} \cite{Veitch:2014wba}, \textsc{PyCBC Inference}
\cite{Biwer:2018osg}, and \textsc{Bilby} \cite{Ashton:2018jfp}.  These packages
have been tested by a number of groups and are well regarded as standard tools.
However, while these tools have passed many robustness tests, they are known to
be computationally intensive. The exact amount of time needed to analyze one
event depends on factors like the duration and frequency of the signal, as well
as features of the specific waveform model. Typical runtimes for
production-level analyses can range from hours to weeks. This expense precludes
iterating quickly on results, launching large scale measurement simulations, or
obtaining results in low latency to inform astronomers for potential followup
in real time.

Additionally, in the coming decade, there are planned upgrades for existing
facilities, as well as plans for next-generation detectors such as the Einstein
Telescope (ET) \cite{Punturo:2010zz} and the Cosmic Explorer (CE)
\cite{LIGOScientific:2016wof}. These upgrades will increase the sensitivity of
the instruments and allow for the detection of more events with a better
signal-to-noise ratio (SNR). The number of events that will be detected in the
coming decade is expected to grow from around a thousand per year to over a
million per year \cite{Baibhav:2019gxm}. This will put a significant strain on
the current PE tools.

% Effort going to new tools development.
In order to address this, there are efforts from multiple groups to speed up
the PE process. This includes methods that employ modern tools such as methods
based on deep learning networks pre-trained on large collections of waveforms
\cite{Dax:2021tsq,Dax:2022pxd}, as well as methods that reduce the
computational expense of classical PE by leveraging our knowledge of GW signals
\cite{Islam:2022afg,Roulet:2022kot}. While these methods are promising avenues
for standard GW problems, particularly for CBCs in GR, they rely on assumptions
that may not hold for analyses involving additional physical effects such as
lensing and deviations from GR, or may use approximations that do not compute
the Bayesian likelihood exactly.

%Basic idea and unique advantage of our tool

In this work, we present a lightweight, flexible, and high performance
framework to infer GW event parameters in a fully-Bayesian analysis. Our
framework implements the following techniques to achieve its performance:
\begin{enumerate}
\setlength{\itemsep}{0pt}
\item differentiable waveform models,
\item normalizing-flow enhanced Markov chain Monte-Carlo (MCMC) sampler,
\item heterodyned likelihood,
\item native support for hardware accelerators.
\end{enumerate}
The main advantage of our framework is it does not rely on specific assumptions
about the problem to achieve its performance. This makes our method extensible
to problems beyond the standard CBC analysis, without sacrificing accuracy for
efficiency.

%Structure of paper

The rest of the paper is structured as follows: we review the basics of PE and
introduce our framework in Sec.~\ref{sec: PE}; we present benchmarking results
on both simulated and real data in Sec.~\ref{sec: Result}; and, finally, we
discuss the implications of this work and directions for future development in
Sec.~\ref{sec: Discussion}.

\section{Gravitational wave parameter estimation}
\label{sec: PE}

\subsection{Likelihood function}
\label{sec:likelihood}

The main objective of PE is to obtain a multidimensional posterior distribution
$p(\mathbf{\theta} \mid d)$ on parameters $\mathbf{\theta}$ given strain
data $d$.  Such probability density represents our best inference of
the source properties, an encodes all relevant information contained in the
observed data.
To compute this object, we use Bayes' theorem to write
\begin{align} \label{eq: bayes}
    p(\theta \mid d) = \frac{\mathcal{L}(d \mid \theta)\pi(\theta)}{p(d)}\, ,
\end{align}
where $\mathcal{L}(d \mid \theta)$ is the likelihood function, $\pi(\theta)$ is
the prior distribution, and $p(d)$ is the evidence. Since the evidence is a
normalization constant that does not depend on the source parameters, it is
often omitted if we are only interested in the posterior distribution. The prior
distribution is often chosen to be some simple distribution, such as uniformly
distributed in the component masses or a Gaussian distribution in the spins, or
it could encode astrophysical information. Assuming the noise is drawn from a
Gaussian process, the log-likelihood for GW data is given by
\begin{align}
    \log{\mathcal{L}(d \mid \theta)} = -\frac{1}{2} \left\langle d-h(\mathbf{\theta}) \mid d-h(\mathbf{\theta})\right\rangle,
\label{eq: loglikelihood}
\end{align}
where $d$ is the observed strain data, $h(\mathbf{\theta})$ is the signal
predicted by a waveform model with a specific set of source parameters $\theta$.
The right hand side of Eq.~\eqref{eq: loglikelihood} can be evaluated in either
the time or frequency domains. For stationary noise, it is computationally
cheaper to compute the likelihood in the frequency domain and the
noise-weighted inner product can be written 
\begin{align}
    % \left<a|b\right> = 4 Re\int \frac{a^*(f)b(f)}{\mathcal{S}_n(f)} df,
    \left\langle a \mid b\right\rangle = 4 \Re \int \frac{a^*(f)b(f)}{\mathcal{S}_n(f)}\, \mathrm{d}f \, ,
\label{eq: innerproduct}
\end{align}
where $\mathcal{S}_n(f)$ is the noise power spectral density (PSD).
In practice, the integral becomes a discrete sum over a finite number of
samples determined by the sampling rate of the detector data and duration of
the observation.

To compute the integral shown in Eq.~\eqref{eq: innerproduct}, we need to
evaluate a chosen waveform model $h(\mathbf{\theta})$ at a number of frequency
sample points. This makes evaluating the likelihood function often the most
computationally intensive part of PE. The most accurate waveform model is
numerical relativity (NR), which obtains waveforms by directly solving the
Einstein equations numerically for a given system. However, depending on the
source parameters, generating one time series of strain can take a day to half
a year, which makes NR prohibitively expensive for PE. To circumvent this
problem, there are several families of waveform ``approximants'', including the
\textsc{IMRPhenom} family \cite{Khan:2015jqa, Garcia-Quiros:2020qpx}, the
\textsc{SEOB} family \cite{PhysRevD.89.061502}, and the NR surrogate
family\cite{Varma:2019csw}. For shorter events, such as a $30-30\ M_{\odot}$
binary black hole, one waveform call could take $10\text{ms}$ to ${\sim}1s$
\kw{verify number in lalsuite}. For longer events, such as a $1.4-1.4\
M_{\odot}$ binary neutron star, the evaluation time could go up to \kw{Fill}.
Since one needs to evaluate the likelihood millions of times during
sampling,\footnote{A typical PE run with \textsc{Bilby} takes ${>}10^6$
likelihood evaluations to converge.} the computational cost in evaluating the
waveform accumulates and is the main reason of the long runtime of GW PE.

\subsection{Heterodyned likelihood}

Since the computational cost of evaluating a waveform model scales linearly
with the number of sample points either in the time or frequency domain, the
computational burden for longer-duration signals is often quite large. To
reduce the computational cost, there are a number of methods to reduce the
number of basis points one would need to compute the likelihood faithfully
\cite{Field:2011mf, Field:2013cfa, Smith:2016qas, Vinciguerra:2017ngf}. In this
work, we use likelihood heterodyning \cite{Cornish:2021lje} (also named
relative binning in \cite{Zackay:2018qdy}).

The idea behind the heterodyned likelihood can be summarized as follows: the
integrand in Eq.~\eqref{eq: innerproduct} is a highly oscillatory function, so
one has to sample the integrand with sufficiently dense sampling to compute the
integral faithfully. The number of sample points needed would be much smaller
if the integrand was smooth. Given a pair of waveform parameters
$\mathbf{\theta}$ and $\mathbf{\theta_0}$ that are close to each other, the
waveforms generated using the pair of parameters are similar to each other,
this means the ratio between the waveforms is a smoothly varying function.
Given a reference waveform $h(\mathbf{\theta_0})$, we can exploit this
similarity between waveforms to reduce the number of sample points needed to
compute the likelihood for the set of $\mathbf{\theta}$ that is similar to
$\mathbf{\theta_0}$. We decompose the integrand into two parts: (1) a highly
oscillatory part that depends only on the reference waveform given by
$\mathbf{\theta}$ and the data, and hence only needs to be evaluated once; and,
(2) a smoothly varying part that depends on the target waveform parameters
$\theta$, that needs to be evaluated for every new likelihood evaluation.
Because the part that depends on the target waveform parameters is smooth, we
can use far fewer sample points to compute the integral with sufficient
accuracy.

One may be concerned by the accuracy of this approximation over the target
parameter space, especially in the region where the generated waveform is
significantly different from the reference waveform. However, given that we are
interested in the most probable set of parameters, if we choose the reference
waveform to be close to the data, the waveforms that are different from the
reference waveform should necessarily also differ significantly from the data.
This means that the likelihood value for these waveforms should be
significantly smaller than the likelihood of the waveforms that are similar to
the reference waveform, and hence will not be relevant for the PE result. In
practice, one will first optimize the likelihood function with full frequency
resolution to obtain the reference waveform parameters, which can be run at a
much lower cost compared to PE. 

We now give a concise description of the implementation of this approach in our
code; for a more extensive derivation of heterodyned likelihood, we refer the
reader to the reference \cite{Zackay:2018qdy}. In the heterodyned likelihood
framework, the two terms involving $h$ obtained by expanding Eq.~\eqref{eq:
loglikelihood} can be approximated as
\begin{subequations} \label{eq: heterodynedlikelihood}
\begin{equation}
    \langle d \mid h \rangle \approx \sum_b \left[ A_0(b)\, r^*_0(h,b) + A_1(b)\, r^*_1(h,b) \right] ,
\end{equation}
\begin{align}
    \langle h \mid h \rangle \approx \sum_b &\left[ B_0(b)\, |r_0(h,b)|^2 + \right. \nonumber \\
    &\left. 2 B_1(b)\, \Re\{r_0(h,b)\, r_1(h,b)\} \right] 
\end{align}
\end{subequations} 
where $b$ denotes the index of a \textit{sparse} set of bins over which the integrand
will be computed; $A_0(b)$, $A_1(b)$, $B_0(b)$, and $B_1(b)$ are the
heterodyning coefficients computed using the data and the reference waveform;
and, finally, $r_0(h,b)$ and $r_1(h,b)$ are the ratios between the target waveform and the
reference waveform at the center of the bin and its first derivative.
For sufficiently fine bins, the ratio between the target waveform
and the reference within a bin can be approximated by linear interpolation,
\begin{align}
r(f) = \frac{h(f)}{h_0(f)} = r_0(h,b) + r_1(h,b)(f- f_m(b)) + \cdots,
\label{eq:definer}
\end{align}
where $b$ is the index of a particular bin, $r_0(h,b)$ and $r_1(h,b)$ are the
value and slope of the ratio at the center of the bin respectively, and
$f_m(b)$ is the center frequency of the bin. Since we have access to both $h(f)$
and $h_0(f)$, we can compute $r_0$ and $r_1$ by evaluating the value of $r(f)$
at the edge of the bin and inverting Eq.~\eqref{eq:definer}.
To evaluate Eq.~\eqref{eq: heterodynedlikelihood}, we need to first choose a
binning scheme, then evaluate the coefficients given the data and the reference
waveform, and at last the ratio between the target waveform and the reference
waveform at the center of each bin.

Considering the phasing of a waveform is denoted by a power series $\Psi(f) =
\sum_i \alpha_i f^{\gamma_i}$, where $\alpha_i$ are some coefficients depending
on the waveform parameters and $\gamma_i$ are powers motivated by post-Newtonian
theory. For example, for the term $\gamma_i = -5/3$, $\alpha_i$ is related to
the chirp mass. The maximum dephasing one can have within a frequency interval
$[f_{\textrm{min}},f_{\textrm{max}}]$ is given by
\begin{align}
    \delta \Psi_{\textrm{max}}(f) = 2\pi \sum_{i} (f/f_{*,i})^{\gamma_i} \textrm{sgn}(\gamma_i),
\label{eq: maxdephasing}
\end{align}
where $f_{*,i} = f_{\textrm{max}}$ for $\gamma_i \geq 0$ and $f_{*,i} =
f_{\textrm{min}}$ for $\gamma_i<0$. Given the relation shown in Eq.~\eqref{eq:
maxdephasing}, we can choose the binning scheme to divide the entire frequency
band of interest into a set of bins such that the maximum dephasing within each
bin is smaller than a certain threshold $\epsilon$, i.e.,
$|\delta\Psi_{\textrm{max}}(f_{\textrm{max}}) -
\delta\Psi_{\textrm{max}}(f_{\textrm{min}})| < \epsilon$. 

The final ingredient we need is the heterodyning coefficients for the data and
the reference waveform on the sparse bins, which are explicitly given by
\begin{align}
    A_0(b) &= 4 \sum_{f \in b} \frac{d(f)h^*_0(f)}{S_n(f)} \Delta f, \\
    A_1(b) &= 4 \sum_{f \in b} \frac{d(f)h^*_0(f)(f-f_m(b))}{S_n(f)} \Delta f, \\
    B_0(b) &= 4 \sum_{f \in b} \frac{|h_0(f)|^2}{S_n(f)} \Delta f, \\
    B_1(b) &= 4 \sum_{f \in b} \frac{|h_0(f)|^2(f-f_m(b))}{S_n(f)} \Delta f.
\end{align}
Note that the sum within each bin should be done with the same sampling rate as
the data, i.e., the same one would do without using the heterodyned likelihood.

% \te{Do we also already here want to make the point that it really helps with memory issues of GPUs if the frequency array isn't too large?}

To obtain a reference waveform, we currently use the \textsc{differential
evolution} algorithm \cite{Storn1997DifferentialE} available in the
\textsc{scipy} package \cite{2020SciPy-NMeth} to find the waveform parameters
which maximize the likelihood.  The reference waveform could also be produced
from trigger parameters precomputed by a search pipeline without additional
computation.

\subsection{MCMC with Gradient-based sampler}
\label{sec:gradient}

Given Eq.~\eqref{eq: loglikelihood} and the prior, one can evaluate the
posterior density function, Eq.~\eqref{eq: bayes}, over the entire parameter
space of interest to obtain the most probable set of values that are consistent
with the data. However, directly sampling this posterior quickly becomes
intractable as the dimensionality of the parameter space increases beyond a few
dimensions. Markov chain Monte Carlo \cite{gelmanbda04} is a common
method employed to generate samples from the target posterior when direct
sampling is not possible.

In MCMC, the posterior distribution is approximated by a Markov chain that eventually
converges to the target distribution \cite{10.1214/aos/1176325750}. The Markov
chain is constructed by iteratively proposing a new point in the parameter space
based on the current location of the chain. The proposed point is accepted with
a probability that is usually set to be proportional to the ratio of the
posterior density evaluated at the proposed point and the current point. The
chain can either accept the proposal and move to the new location, or reject the
proposal and stay at the current location. This process is repeated until the
chain converges to the target distribution. The samples generated by the chain
are then used as a fair sample to estimate the quantities of interest, such as the mean and
credible intervals of the source parameters. In practice, since we do not know
the target distribution ahead of time, the MCMC process is usually repeated
until a certain criterion is met, such as a Gelman-Rubin convergence statistic
\cite{10.2307/2246093} lower than certain threshold, or simply after a fixed number
of iterations.

Compared to direct sampling, MCMC algorithms only explore regions that are
highly probable, thus reducing the computational cost by not wasting resources
in regions where it is unlikely to generate the observed data. However, MCMC
algorithms come with their own set of issues. To illustrate what difficulties
MCMC may face, we can examine one of the most vanilla MCMC algorithms: the
Metropolis-Hastings algorithm with a Gaussian kernel. Starting at some initial
point, one can draw a proposed point from a Gaussian transition kernel, defined
as
\begin{align}
    q(\mathbf{x},\mathbf{x_0})= \mathcal{N}(\mathbf{x} \mid \mathbf{x_0},\mathbf{C}),
\end{align}
where $\mathbf{x_0}$ is the current location of the chain, $\mathbf{x}$ is the
proposed location, and $\mathbf{C}$ is the covariance matrix of the Gaussian. In
the simplest case, we can pick $\mathbf{C}$ to be a diagonal matrix with a
constant value, which corresponds to an isotropic Gaussian center around the
current location and with a fixed variance. The acceptance criterion is
defined as
\begin{align}
\alpha(\mathbf{x},\mathbf{x_0}) = \min\left(1,\frac{p(\mathbf{x})q(\mathbf{x_0},\mathbf{x})}{p(\mathbf{x_0})q(\mathbf{x},\mathbf{x_0})}\right).
\label{eq:Gaussian_acceptance}
\end{align}

We can see from Eq.~\eqref{eq:Gaussian_acceptance} that the acceptance rate is
proportional to the fraction of volume where the posterior density at the
proposed location is higher than the current location within the Gaussian
transition kernel. If we choose the variance of the transition kernel to be too
large, this fraction will be small hence the acceptance rate will be poor. On
the other hand, if one chooses the variance to be too small, nearby samples
will be correlated, and it will take a long time for the chain to wander. In
both cases, the efficiency in constructing the chain with a target number of
independent samples is suboptimal. Consequently, there is often a tuning
process before we run the MCMC algorithm to find the optimal settings for the
algorithm (in this example, the variance of the Gaussian) to ensure the best
possible performance.

However, as we often deal with high-dimensional problems, even the optimally
tuned Gaussian transition kernel does not guarantee good performance. In order
to have a reasonable acceptance rate, the variance of the Gaussian has to be
smaller in a higher dimensional space, which means that the transition kernel will generally make smaller and smaller steps as we increase the dimensionality of the problem
\cite{2017arXiv170102434B}.

Transition kernels that leverage gradient information of the target
distribution can help to address this issue of shortening steps in a high
dimensional space. Instead of proposing a new point by drawing from a Gaussian,
one can use the gradient evaluated at the current location to propose a new
point, so that the evolution of the chain is preferentially directed to regions
of higher probability. For example, Metropolis-adjusted Langevin algorithm
(MALA) \cite{10.2307/2346184} place a unit Gaussian at the tip of the gradient
vector at the current position,
\begin{align}
    \mathbf{x} = \mathbf{x_0} + \tau \nabla\log{p(\mathbf{x_0})} + \sqrt{2\tau}N(0,\mathbf{I}),
\end{align}
where $\tau$ is a step size chosen during the tuning stage. Compared to a
Gaussian centered at the current location, the MALA transition kernel is more
likely to propose a point in the higher posterior density region because of
the gradient term, which helps boost the acceptance rate.

While transition kernels that use gradient information can help improve the
acceptance rate, computing the gradient of the posterior density function
introduces an additional computational cost, which is not necessarily
beneficial in terms of sampling time. If one wants to compute the gradient
through finite differencing, the additional computational cost goes as at least
${\sim} \mathcal{O}(2n)$, where $n$ is the dimension of the problem. On the
other hand, automatic differentiation schemes like \textsc{Jax} allow us to
compute the gradient of the likelihood function with respect to the parameters
through automatic differentiation, which gives the gradient information down to
machine precision at around the same order of time compared to evaluating the
posterior itself. Thus, having access to gradient information through automatic
differentiation is crucial to making gradient-based transition kernels
favorable in terms of computing cost.

\subsection{Normalizing Flow enhanced sampling}
\label{sec:flow}

% Problem with just HMC or MALA
While gradient-based samplers have been shown to outperform gradient-free
algorithms in many practical applications, there remain classes of problems that
most gradient-based samplers do not solve well. For example, first-order
gradient-based algorithms struggle with target distributions that exhibit
locally-varying correlations, since they assume a single mass matrix that does
not depend on the location of the chain by construction
\cite{2017arXiv170102434B}. \footnote{Sampling algorithms that use the
information of higher order derivatives such as manifold-MALA and Riemannian-HMC
\cite{RMHMC} can in principle handle local correlations in the target
distribution; however, they often encounter instabilities when used in real-life
applications, so their use is a rare practice.} Another example is multi-modality:
if there are multiple modes in the target distribution, individual chains will
likely be trapped in one mode and take an extremely long time to transverse
between the modes \cite{2018arXiv180803230M}.  This means that the relative
weights between modes will take much longer to estimate than the shape of each
mode.

% Long burn-in too
Moreover, before we can use the sampling chain to estimate the posterior
quantities we care about, the sampler often needs to first find the most
probable region in the target space (known as the \emph{typical set}); this is
a common process often referred to as ``burn-in'' in the literature. As a
consequence, one would discard a certain amount of data generated from the
beginning of the sampling process, and only use the later part of the chain to
estimate the quantities of interest. The burn-in phase of a gradient-based
sampler is often as long as the sampling phase, which means that a good portion
of the computation is not directly devoted to estimating the target quantities.

% Crux of normalizing flow
All the above issues can be mitigated by normalizing flows.  Normalizing flows
is a technique based on neural networks that aims at learning a mapping from a
simple distribution, such as a Gaussian, to a complex distribution, often given
in the form of samples \cite{2019arXiv190809257K, 2019arXiv191202762P}. Once
the network is trained, one can evaluate the probability density of the complex
distribution and sample from it very efficiently, by first evaluating the
simple distribution and then applying the learned mapping. The core equation
of normalizing flows is the coordinate transformation of probability
distributions via a Jacobian, as given by
\begin{align}
    p_x(X) = p_z(Z) \left| \frac{\partial f}{\partial z}\right|^{-1},
\end{align}
where $p_x(X)$ is the complex target distribution, $p_z(Z)$ is the simple
latent distribution and $f$ is an invertible parameterized transform that
connects the two distributions, $x = f(z)$, to be learned by the normalizing
flow. For a detailed discussion of the algorithm, we refer the readers to
\cite{2019arXiv190809257K, 2019arXiv191202762P}.

% The flowMC algorithm
Working in tandem, gradient-based MCMC and normalizing flows can efficiently
explore posteriors with local and global correlations, as well as multiple
separate modes.  The scheme relies on iteratively using draws from the
gradient-based MCMC to train a normalizing flow, which is then itself used as a
proposal for another stage of MCMC sampling.

Concretely, we begin by producing initial training data for the normalizing
flow by running multiple independent chains of the gradient-based algorithm for
a fixed number of steps.  From the resulting pool of samples, the normalizing
flow can begin to learn the landscape of the target distribution.
However, since the independent chains contain the same number of samples, the
relative weight assigned to each chain will not represent the true target
distribution (e.g., the relative importance of separate modes will not be
correctly calibrated). This is mitigated by a second stage of gradient-based
MCMC sampling that uses the distribution learned by the normalizing flow as a
\textit{proposal}.

Given a trained normalizing flow model, we can generate the proposed jump in
the target space by sampling from the latent distribution $z \sim p_z(Z)$,
usually a Gaussian, and then pushing it through the learned map given by the
normalizing flow model $x=f(z)$.  The acceptance criterion is then set to be
\begin{align} \label{eq:flow-proposal}
    \alpha(\mathbf{x},\mathbf{x_0}) = \min \left[ 1, \frac{\hat{\rho}(\mathbf{x_0})\rho_*(\mathbf{x})}{\hat{\rho}(\mathbf{x})\rho_*(\mathbf{x_0})}\right],
\end{align}
where $\hat{\rho}$ is the probability density estimated by the normalizing flow
model, $\rho_*$ is the probability density evaluated using the target function,
and $x_0$ is the current position.

From Eq.~\eqref{eq:flow-proposal}, we can see that the flow distribution is the
target distribution when the accepting probability is 1. When the normalizing
flow model has not converged to the target distribution, only a portion of the
proposed jumps will be accepted. This means an MCMC process using the
normalizing flow model as the proposal distribution can adjust the
normalization across different regions of the target parameter space by
rejecting jumps into less likely regions. The training and sampling are then
repeated until certain criteria, at each step combining global and local MCMC
sampling which respectively do and do not use the normalizing flow as proposal.

Note that every time we retrain the network, we are breaking
the Markov properties since we are changing the proposal distribution. To
produce final samples that can be used to estimate target quantities, one has to
freeze the normalizing flow model and not retrain during the final sampling
phase in order to satisfy the detailed balance condition. We use the package
\textsc{flowMC} \cite{2022arXiv221106397W}. The pseudocode of the algorithm is
given in Algorithm \ref{alg:cap}.

\begin{algorithm}
\caption{\textsc{flowMC} pseudocode}\label{alg:cap}
\KwIn{initial position $ip$}
\Parameters{number of training loops $nt$, number of production loops $np$}
\Variables{current chain $cc$, current position $cp$, current NF parameters $\Theta$}
\KwResult{$chains$}
$cp \leftarrow ip$\\
\tcc{Training loop}
    \For{$i<nt$}{
        $cc, cp \leftarrow LocalSampling(cp)$\\
        $\Theta \leftarrow TuneNF(cc)$\\
        $chains, cp \leftarrow GlobalSampling(cp, \Theta)$ \\
        $cc \leftarrow Append(cc, chains)$
    }
\tcc{Production loop}
    \For{$i<np$}{
        $c_{local}, cp \leftarrow LocalSampling(cp)$\\
        $c_{global}, cp \leftarrow GlobalSampling(cp, \Theta)$ \\
        $chains \leftarrow Append(chains, c_{local}, c_{global})$
    }

\Return{$chains$}
\end{algorithm}

\subsection{Accelerators}
\label{sec:accelerators}

Modern hardwares such as graphics processing units (GPUs) and
tensor processing units (TPUs) are designed to execute large-scale dense
computation. They are often much more cost-efficient than using many central
processing units (CPUs) when it comes to solving problems that can be benefited
from parallelization.  The downside of these accelerators compared to CPUs is
that they can only perform a more restricted set of operations and are often
less performant when dealing with serial problems. Parameter estimation with
MCMC is a serial problem since each new sample generated from a chain depends
on the last sample in the chain. This means that naively putting the problem on
an accelerator is more likely to harm performance than improve it.

% More sample helps training normalizing flow
Yet, in our work, the use of accelerators provides two independent perks that
tremendously benefit the parameter estimation process. First, using
accelerators allow us to run many independent MCMC chains simultaneously, which
benefits the training of the normalizing flow. Since we generate the data we
use to train the normalizing flow on the fly, the more independent data we can
feed to the training process, the higher chance the normalizing flow can learn
a reasonable representation of the global landscape of the target distribution.
If we only used a small number of chains, we would be limited to the correlated
samples from each chain and we would have to run more sequential steps to get
the same amount of independent samples---with more chains the problem becomes
parallelizable and we can obtain the same number of training samples sooner. In
other words, being able to use many independent chains helps the normalizing
flow learn the global landscape faster in wall time.

% GPU helps packing a shit ton of waveform evaluation
Another benefit of accelerators is the parallel evaluation of waveforms. Since
the waveform model we use can be evaluated at any given time or frequency
independently, this means computing a waveform can be trivially parallelized
over frequency bins. Together with the heterodyned likelihood, we can evaluate
the likelihood at $\mathcal{O}(10^7)$ different locations on an \textsc{Nvidia}
A100 GPU.  The high throughput of likelihood evaluations unlocks the potential
of the \textsc{flowMC} sampling algorithms.

\section{Result}
\label{sec: Result}
\subsection{Injection-recovery test}

To demonstrate the robustness of our pipeline, we use it to recover the
parameters of a set of simulated signals. We create a set of simulated signals
and inject them into simulated stationary Gaussian noise.
Then we run our pipeline on the simulated data, and determine the credible
interval at which the true parameters of the injected signals are recovered. From the set of
credible values, we can check whether the true value lies within a certain
credible interval at the expected frequency: if our pipeline is working as expected, we
should find the true parameters lie within $x\%$ credible interval $x\%$ of the
time, e.g., the true value should lie within the $50\%$ credible interval $50\%$
of the time. In other word, the percentiles of the true parameters should be
uniformly distributed. Deviation from this behavior suggests the pipeline is
either over-confident or too conservative \cite{Cook2006,Talts2018}.

% State the distribution of injected population
We sample 1200 events from the distribution of parameters detailed in Table
\ref{tab:parameters}; the same distributions are used as the prior in the PE
process.  We simulate signals over 16 seconds of data, with a minimum frequency
cutoff of 30 Hz and a sampling rate of 2048 Hz.  We draw noise from the design
PSDs for the LIGO Hanford, LIGO Livingston
(\textsc{SimNoisePSDaLIGOZeroDetHighPower}) and Virgo
(\textsc{SimNoisePSDAdvVirgo}) detectors \cite{}.
For both injection and recovery, we make use of the \textsc{IMRPhenomD} waveform
\cite{Khan:2015jqa} via the fully-differentiable implementation presented in the
\textsc{ripple} package \cite{}.

\begin{table*}[hbt!]
    \begin{center}
    \begin{tabular}{ l l l l l }
    \hline
    \hline
    Parameter &  Description & Injection & GW150914 & GW170817\\
    \hline

    $M_c$ & chirp mass $[M_\odot]$& $[10, 50]$ & $[10,80]$ & $[1.18,1.21]$ \\
    $q$ & mass ratio & $[0.5, 1]$ & $[0.125,1]$ & $[0.125,1]$ \\
    $\chi_1$ & primary dimensionless spin& $[-0.5, 0.5]$ & $[-1,1]$ & $[-0.3,0.3]$ \\
    $\chi_2$ & secondary dimensionless spin & $[-0.5, 0.5]$ & $[-1,1]$ & $[-0.3,0.3]$ \\
    $d_L$ & luminosity distance $[\textrm{Mpc}]$ & $[300, 2000]$ & $[0, 2000]^\dag$ & $[1, 75]^\dag$ \\
    $t_c$ & coalescence time $[\textrm{s}]$& $[-0.5, 0.5]$ & $[-0.1, 0.1]$ & $[-0.1, 0.1]$ \\
    $\phi_c$ & coalescence phase & $[0, 2\pi]$ & $[0, 2\pi]$ & $[0, 2\pi]$ \\
    $\cos{\iota}$ & cosine of inclination angle & $[-1, 1]$ & $[-1, 1]$ & $[-1, 1]$ \\
    $\psi$ & polarization angle & $[0, \pi]$ & $[0, \pi]$ & $[0, \pi]$ \\
    $\alpha$ & right ascension & $[0, 2\pi]$ & $[0, 2\pi]$ & $[0, 2\pi]$ \\
    $\sin{\delta}$ & sine of declination & $[-1, 1]$ & $[-1, 1]$ & $[-1, 1]$ \\

    \hline
    \hline
    \end{tabular}
    \caption{Prior ranges for parameters varied in the injection-recovery
    test, as well as the GW150914 and GW170817 analyses. All priors are uniform over the ranges shown, except for the luminosity distance prior in the GW150914 and GW170817 analyses ($^\dag$) for which we apply a prior unform in comoving volume. The coalescence time refers to a shift relative to the geocenter trigger time, and $M_c$ refers to the redshifted (detector-frame) chirp mass.}
    \label{tab:parameters}
    \end{center}
\end{table*}

% Show injection pp plot
We summarize the result of this injection-recovery campaign in
Fig.~\ref{fig:ppplot}. This shows the cumulative distribution over injections
of the quantile at which the true value lies in the marginalized distribution
of each parameter. The shaded band denotes the 95\%-confident variation
expected from draws from a uniform distribution with the same number of events.
We can see that most of the measured curves lie within this band, showing that
our inference results agree well with a uniform distribution. There is a small
deviation from a uniform distribution for the secondary spin $\chi_2$, which is
not alarming given that we are computing the quantile for 11 parameters.  This
is what we expect if our pipeline is working as expected.

To further quantify how well our result agrees with a uniform distribution, we
can compute the Kolmogorov-Smirnov $p$-values for each marginalized
distribution \cite{}.  If this $p$-value were is low (with a threshold often
chosen to be $p = 0.05$), then our result could be in tension with a uniform
distribution. The $p$-values obtained for each parameter are shown in the
legend of Fig.~\ref{fig:ppplot}.  Most of them are well above the $p = 0.05$
threshold, except for $\chi_2$, which is mildly below the threshold. Once
again, assuming these $p$-values are drawn from a uniform distribution, given
11 draws (the number of parameters in our inference), it is not abnormal to
have one of the parameters lying slightly outside the threshold. To assess
whether this is expected, we can compute the combined $p$-value for these 11
parameters, and find it to be $p = 0.47$.  This shows our inference pipeline
performs properly on simulated data at a similar level as standard tools
\cite{Veitch:2014wba,Romero-Shaw:2020owr}.

\begin{figure}
    \script{ppplots.py}
    \includegraphics[width=0.99\linewidth]{figures/ppplot.pdf}
    \caption{Cumulative distribution of the quantile of which the true value
    lies for each marginalized distribution. The shadow band denotes the 95\%
    credible interval drawn from a uniform distribution with the same number of
    events as the injection campaign. The legend shows the p-values for each
    marginalized distribution.}
    \label{fig:ppplot}
    \end{figure}

\subsection{Real event parameter estimation}

To demonstrate the performance of our parameter estimation pipeline, we apply
it to two real LIGO-Virgo events: GW150914 and GW170817. We use the priors
shown in Table \ref{tab:parameters}, and take 4 s of data sampled at 2048 Hz
for the GW150914 analysis, and 128 s of data sampled at 4096 Hz for the
GW170817 analysis; strain data and PSDs for both events are fetched from GWOSC
\cite{GWOSC}. For our specific choice of sampler settings, we produce
${\sim}$2500 and 3500 \emph{effective samples}\footnote{Effective
samples here refers to the number of independent samples, which is the total
number of generated samples divided by their correlation length; we compute the
effective sample size using
\textsc{arviz} \cite{arviz_2019},\url{https://python.arviz.org/en/stable/api/generated/arviz.ess.html}.
} for GW150914 and GW170817 respectively. Running on an \textsc{Nvidia} A100
GPU, the wall time for both events is around 10 minutes. Most of this
time is spent on just-in-time (JIT) compilation of the code; the actual
sampling time is only ${\sim}150$ s. We pre-compute the reference waveform
parameters used in hetetrodyne likelihood for the two events, the time for
which is omitted in the \texttt{wall} time calculation.  The chain data and the
analysis scripts which generate the chains can be found in 
% \kw{\variable{data/} and \variable{script}}.

\begin{figure}
    \script{plotEvents.py}
    \includegraphics[width=0.99\linewidth]{figures/GW150914.pdf}
    \caption{
      GW150914 posterior computed by our code (blue) and \textsc{Bilby} (orange).
    }
    \label{fig:GW150914}
\end{figure}

\begin{figure}
    \script{plotEvents.py}
\includegraphics[width=0.99\linewidth]{figures/GW170817.pdf}
\caption{
    GW170817 posterior computed by our code (blue) and \textsc{Bilby} (orange).
}
\label{fig:GW170817}
\end{figure}

For comparison, we produce equivalent runs with \textsc{Bilby}, using the same
exact data and priors.  We use the \textsc{dynesty} sampler
\cite{2020MNRAS.493.3132S,dynesty}, with 1000 live points and other settings as
shown in \cite{release}.  We carry out these runs using \textsc{parallel Bilby}
(\textsc{pBilby}) \cite{Smith:2019ucc} to distribute the computation over 400
Intel Skylake CPUs for each event.  For the specific settings chosen, the
wall-time duration of each run was ${\sim}2$ h for GW150914 and ${\sim}1$ day
for GW170817.

% Describe the distance between flowMC result and Bilby result using
% Jensen-Shannon divergence, scipy has a function to compute that in
% scipy.spatial.distance

Figs.~\ref{fig:GW150914} and \ref{fig:GW170817} show that our posteriors are
consistent with those produced by \textsc{Bilby}.  For a quantitative
comparison, we compute the Jensen-Shannon divergence between our code and
\textsc{Bilby} for the marginalized distribution for each parameter. The
Jensen-Shannon divergence (JSD) is a symmetric measure of the distance between
two probability distributions, with a value of 0 indicating identical
distributions and a value of $\ln{2} \textrm{nat}$ representing the maximum
possible divergence between two distributions. The JSD values for the two events
are shown in Table \ref{tab:JSD}. The maximum JSDs for GW150914 and GW170817 are
$0.0171$ and $0.0166$, and the mean JSDs are $0.0062$ and $0.0031$,
respectively. The JSD values are comparable to those reported in
\cite{Romero-Shaw:2020owr}, which show our code agrees with existing tools.

\begin{table}[hbt!]
    \begin{center}
    \begin{tabular}{ l l l}
    \hline
    \hline
    Parameter & GW150914 & GW170817\\
    \hline
        $M_c$ & $\textbf{0.0171}$ & $\textbf{0.0166}$ \\
        $q$ & $0.0036$ & $0.0004$ \\
        $\chi_1$ & $0.0023$ & $0.0006$ \\
        $\chi_2$ & $0.0098$ & $0.0004$ \\
        $d_L$ & $0.0009$ & $0.0076$ \\
        $\phi_c$ & $0.0003$ & $0.0024$ \\
        $\iota$ & $0.0026$ & $0.0045$ \\
        $\psi$ & $0.0005$ & $0.0034$ \\
        $\alpha$ & $0.0007$ & $0.0242$ \\
        $\delta$ & $0.0020$ & $0.0017$ \\
    \hline
    \hline
    \end{tabular}
    \caption{JSD values for the marginalized distribution of each parameter for
    GW150914 and GW170817 between our code and \textsc{Bilby}. The bold values
    indicate the parameters with the largest JSD.}
    \label{tab:JSD}
    \end{center}
\end{table}



\section{Discussion}
\label{sec: Discussion}

\subsection{Comparison to other approaches}

% Compared to other works
There have been several recent efforts to speed up parameter estimation of
gravitational wave, relying on techniques ranging from efficient
reparameterizations \cite{Islam:2022afg,Roulet:2022kot} to deep learning
\cite{Dax:2021tsq,Dax:2022pxd}. While all of these methods can achieve
minutes-scale parameter estimation with high fidelity, our approach possesses
unique strengths, and may complement some of those other techniques.

% Stephen's work
In contrast to \cite{Dax:2021tsq,Dax:2022pxd}, we do not require pre-training
the neural network on a large collection of waveforms and noise realizations.
This means that our algorithm can be immediately deployed as soon as new
waveform models and noise models are available. Furthermore, our method is at
its core an MCMC algorithm, meaning it inherits the merit of convergence
measures in MCMC. As we are only using the normalizing flow as a proposal
distribution, and the normalizing flow is trained jointly with a local sampler,
we do not risk overfitting since our training data is being generated on the
fly and is always approaching the target distribution. In this sense, we do not
introduce potential extra systematic errors to the inference results.

While our pipeline uses samples generated by the local sampler for training,
one could also supply a pre-trained normalizing flow to our pipeline to bypass
the training stage. This would have the advantage of further reducing the total
runtime; however, it could introduce systematic bias in the inference result if
the pre-trained network is not able to capture the complexity presented in the
data.

%Tousif's work
In contrast to \cite{Islam:2022afg,Roulet:2022kot}, we do not rely on
handcrafted reparameterizations of the coordinate systems used for sampling. If
a useful reparameterization scheme is known ahead of time, it could trivially
be implemented within our pipeline, potentially easing convergence. However,
handcrafted reparameterizations rely on specific assumptions about the targeted
signal, which cannot always be generalized beyond specific applications. On the
other hand, within our pipeline, the normalizing flow effectively discovers
reparameterizations that ease sampling automatically without \emph{a priori}
knowledge of the structure of the problem. In general, the transformation
discovered by the normalizing flow will only be approximate and hence not as
efficient as an explicit reparameterization of the problem; yet, our approach
applies to a much border class of problems where clever reparameterizations are
not known ahead of time, such as parameter estimation with precessing
waveforms, calibration parameters, testing GR, and multi-event joint inference.

It is always beneficial to reparameterize if the reparameterization is known
ahead of time. For the class of problems treated in
\cite{Islam:2022afg,Roulet:2022kot}, we can incorporate those
reparameterizations directly into our MCMC pipeline to reduce the complexity of the
problem, hence speeding up the training phase. If there are limitations to the
reparameterization that mean it cannot properly encompass part of the target
posterior, the normalizing flow should still be able to learn to produce
accurate samples efficiently.

The two avenues discussed here (machine learning and reparameterizations)
represent two orthogonal directions one can take in building next generation PE
tools. On the one hand, there are modern techniques such as deep learning that
are very flexible and powerful, but rely on having highly robust training data.
On the other hand, there are traditional tools that make use of our
understanding of the underlying physics to simplify the problem, which relies
on having good intuition of the problem. Both approaches rely on having a
somewhat reasonable prior to approach the problem. The main difference between
methods used in industrial products and scientific problems is scientific
problems often try to address questions one may have not been answered before,
hence it is supposed to eventually depart from our prior knowledge, meaning its
desirable to design methods that generalize beyond the current state of
knowledge robustly. Our work utilizes both reparameterization and machine
learning, yet our method can be trivially extended to problems beyond standard
GW analysis that reparameterization or deep learning alone may have trouble
dealing with. Beyond efficiency, we believe such flexibility and robustness are
crucial for building scientific tools.

\subsection{Future development}

% Additional traits for having differentiable waveforms
We are currently working on a number of improvements and extensions to our
current infrastructure. While the \textsc{IMRPhenomD} waveform approximant is a
reasonable start, it lacks some qualitative features that other
state-of-the-art models have, such as precession, subdominants moments of the
radiation, and eccentricity. It also has a higher mismatch with reference
numerical relativity waveforms compared to more recent waveform models.
Currently, we are working on building differentiable implementations of
\textsc{IMRPhenomPv2} \cite{Khan:2018fmp}, the precessing successor to
\textsc{IMRPhenomD}, as well as the numerical relativity surrogate waveforms,
including \textsc{NRSur7dq4} \cite{Varma:2018mmi}. Going forward, we expect the
use of autodifferentiation environments like \textsc{Jax} \cite{} to become
more prevalent in the waveform development community, increasing the number of
differentiable waveform models available. This would not only be beneficial for
parameter estimation, but also for a number of other applications such as
Fisher matrix computations, template placement and calibrating waveforms to
numerical relativity \kw{cite}.

% Access to higher dimensional problem
While standard CBC analyses go up to 17 dimensions, non-standard GW PE problems
can have more parameters, which could potentially lead to more complicated
geometry in the target posterior that is hard to reparameterize. For example,
\cite{LIGOScientific:2021sio} introduces 10 extra parameters controlling
deviations in the post-Newtonian coefficients predicted in GR. On top of the
increase of dimensionality, these parameters often introduce non-trivial
degeneracies such as local correlation and multi-modality. Therefore, currently
testing GR is limited in practice to varying these modifications one at a time,
partially due to the bottleneck in the sampler.  Given the gradient-based and
normalizing flow-enhanced sampler, our code shows promise in tackling this
problem.

% Other detectors configuration
Our current code can perform parameter estimation for any combination of
ground-based detectors detectors, under the assumption that signals are
transient and their wavelength is short. The first condition guarantees that
the effect of Earth's rotation can be ignored when computing antenna patterns,
while the second means that we can treat the antenna patterns as frequency
independent constants. These assumptions break for next-generation detectors,
whether on Earth or in space, like Cosmic Explorer, Einstein Telescope and
LISA; differentiable implementations antenna patterns for those detectors is
work in progress.

% More features such as marginalization
Furthermore, our current implementation is minimal and we do not make use of
most standard ``tricks'' to accelerate sampling.  In particular, we do not
incorporate (semi)analytic marginalization schemes over parameters such as
time, phase, and calibration parameters \cite{2019PASA...36...10T}. Because of
the already reasonable performance of our sampler thanks to hardware
accelerators, time and phase marginalization is not crucial, as the performance
of our implementation is not significantly impacted by having two extra
dimensions. Nevertheless, having fewer parameters cannot hurt in the future, so
we are looking into incorporating analytic marginalization schemes in our code
as well.

% JIT overhead
When it comes to wall time, the just-in-time compilation of our code is the
current limiting factor.  While \textsc{Jax}'s JIT compilation drastically
quickens likelihood evaluations, it comes with a significant compilation
overhead before the first evaluation. We observe that the compilation time
depends on the device where the code is run; this is expected since
\textsc{Jax} leverages the \textsc{Accelerated Linear Algebra} (XLA) compiler
to take advantage of hardware accelerators, which means that \textsc{Jax} needs
to compile the code for each specific device according to its architecture. On
an \textsc{Nvidia} A100 GPU, the compilation overhead could go up to 8 minutes
for our current waveform. Meanwhile, for the cases we have studied, the time
needed to obtain converging sampling on an A100 is about 2-3 minutes. This
means the compilation overhead dominates the wall-clock time of our current PE
runs. To maximize the potential of our code, we are looking into ways to reduce
the compilation overhead or to cache the compilation results to avoid paying
the compilation overhead for every event.

% Faster maximum likelihood finder
Besides compilation, there is in principle also overhead from finding the
reference waveform used in for heterodyning the likelihood. Since the
\textsc{differential evolution} algorithm we currently use has not been
implemented in \textsc{Jax}, and the \textsc{Jax} waveform we use is not
compatible with the parallelization scheme in the \textsc{scipy} library,
maximizing the likelihood currently takes us around 5 minutes \kw{Benchmark it
in the code and make this number precise.} for GW170817. There are two ways to
reduce this time.

First, we can explore a different optimization strategy that
takes full advantage of the strengths of our pipeline, in particular the
differentiability of our likelihood and the possibility to evaluate many
waveforms in parallel with a GPU. Particle swarm \cite{7869491} and stochastic
gradient descent methods \cite{10.5555/304710.304720} are promising candidates
we are investigating.

Second, we may marginalize extrinsic parameters to reduce the dimensionality of
the optimization problem. Currently, we simultaneously maximize all 11 CBC
parameters in our problem numerically, which is unnecessary. There are
long-existing, efficient maximization schemes for extrinsic parameters, such as
the merger time and phase, which can find the corresponding maximum likelihood
waveform much more efficiently when compared to differential evolution. We
expect implementing these schemes will reduce the time needed
to find the reference waveform parameters by fixing the extrinsic parameters
and by reducing the dimensionality of the optimization problem.

% Multi-device scaling
Finally, one important aspect of modern computing is scalability, meaning it is
generally favorable if one can simply put more computing units in the same
problem and reduce the wall time. In our case, this means that we would like to
use more than one GPU for the same PE process. More GPUs can help in the
following ways: first, more GPUs means we can run more independent chains at
the same time, which can generate more samples faster; second, and more
importantly, as shown in this work and \kw{cite flowMC?}, more independent
chains also help with reducing the burn-in time. Parallelizing over the number
of chain dimension is trivial and does not require much change to the current
infrastructure. Additional GPUs can also help by enabling faster training of
larger flow models. While the training time is not the biggest bottleneck given
the flow model used in this study, more GPUs means we can increase the
bandwidth of the flow model by increasing its size while keeping the training
time the same. This would help capture more complex geometries in the target
space, which can lead to better convergence in general.

\section{Conclusion}

% Brief summary
In this work, we presented a PE pipeline for GW events that is efficient,
flexible and reliable. Our package brings together a number of innovations,
including differentiable waveform models, likelihood heterodyning, and
normalizing-flow enhanced gradient-based sampling. We tested the robustness of
our pipeline, currently built upon \textsc{ripple} and \textsc{flowMC}, on a
set of 1200 synthetic GW events, showing it is robust, unbiased and efficient
enough to handle the large catalogs of detections that will be available in the
near future. We also show that our pipeline can estimate the parameters of
GW150914 and GW170817 in \kw{Quote speed}, demonstrating the potential of our
implementation on real data.

\section{Acknowledgements}
We thank Will M.~Farr, Aaron Zimmerman, Daniel Foreman-Mackey and Marylou Gabri\'e for helpful discussions.
The Flatiron Institute is a division of the Simons Foundation, supported through the generosity of Marilyn and Jim Simons.
This material is based upon work supported by NSF's LIGO Laboratory which is a major facility fully funded by the National Science Foundation.
This research has made use of data or software obtained from the Gravitational Wave Open Science Center (gw-openscience.org), a service of LIGO Laboratory, the LIGO Scientific Collaboration, the Virgo Collaboration, and KAGRA. LIGO Laboratory and Advanced LIGO are funded by the United States National Science Foundation (NSF) as well as the Science and Technology Facilities Council (STFC) of the United Kingdom, the Max-Planck-Society (MPS), and the State of Niedersachsen/Germany for support of the construction of Advanced LIGO and construction and operation of the GEO600 detector. Additional support for Advanced LIGO was provided by the Australian Research Council. Virgo is funded, through the European Gravitational Observatory (EGO), by the French Centre National de Recherche Scientifique (CNRS), the Italian Istituto Nazionale di Fisica Nucleare (INFN) and the Dutch Nikhef, with contributions by institutions from Belgium, Germany, Greece, Hungary, Ireland, Japan, Monaco, Poland, Portugal, Spain. The construction and operation of KAGRA are funded by Ministry of Education, Culture, Sports, Science and Technology (MEXT), and Japan Society for the Promotion of Science (JSPS), National Research Foundation (NRF) and Ministry of Science and ICT (MSIT) in Korea, Academia Sinica (AS) and the Ministry of Science and Technology (MoST) in Taiwan.

\bibliography{bib}

\end{document}
